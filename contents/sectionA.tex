% !Mode:: "TeX:UTF-8"

\section{研究的基本内容,拟解决的主要问题}


\subsection{插入项目符号}

多智能体系统在多方面多领域得到了广泛的应用:
\begin{itemize}
  \item 军事
  \item 政治
  \item 历史
\end{itemize}

\subsection{插入项目编号}

%% enumerate看效果\Alph*,\alph*,\Roman*,\roman*,\arabic*
多智能体系统的分类:
\begin{enumerate}[label=\alph*)]   
  \item 同构多智能体系统
  \item 异构多智能体系统
\end{enumerate}




\subsection{公式的对齐与引用}

\subsection{安装mathtype}
安装mathtype并根据下图完成配置(图\ref{fig_mathtype1}所示)。
 \begin{figure}[!htb]
  \centering
  \includegraphics[width=1\textwidth]{mathtype配置}
  \caption{mathtype相关配置.}
  \label{fig_mathtype1}
\end{figure}

\subsection{插入带编号的公式及不带编号的公式}
在mathtype编辑公式,并从mathtype直接复制到latex,然后进一步修改。

\textcolor{red}{在文字段落中嵌入公式},此时需用到\$符号。下面是详细步骤,首先从mathtype中直接复制过来,不做任何修改,直接编译效果如下
\[{p_{ij}}(t) = {p_j}(t) + {v_{ij}}(t)\]
\textcolor{red}{如果嵌入到一段文字中},需要去掉 \verb|\[|以及\verb|\]|符号,然后用\$包起来,效果是${p_{ij}}(t) = {p_j}(t) + {v_{ij}}(t)$。

如果不嵌入在一段文字中,让公式单独成行,并编号,可以采用下列步骤。
下面公式是直接复制过来,未加任何修改的编译效果。
\[\begin{array}{l}
V(k) \ge \mathop {\min }\limits_{i \in {\cal V}} {\pi _i}(k)\sum\limits_{i = 1}^N {{{({x_i}(k) - {\pi ^T}(k)x(k))}^2}} \\
 \ge \frac{1}{2}\mathop {\min }\limits_{i \in {\cal V}} {\pi _i}(k){(\mathop {\max }\limits_{i \in {\cal V}} {x_i}(k) - \mathop {\min }\limits_{i \in {\cal V}} {x_i}(k))^2}\\
 \ge \frac{1}{2}\mathop {\min }\limits_{i \in {\cal V}} {\pi _i}(k)(\mathop {\max }\limits_{i \in {\cal V}} {x_i}(k)\\
 \ge \frac{1}{2}\mathop {\min }\limits_{i \in {\cal V}} {\pi _i}(k)(\mathop {\max }\limits_{i \in {\cal V}} {x_i}(k)
\end{array}\]



首先需要去掉 \verb|\[|以及\verb|\]|符号,然后用\verb+\begin{equation}以及\end{equation}+来替换。
\begin{equation}\label{system1}
  \begin{array}{l}
V(k) \ge \mathop {\min }\limits_{i \in {\cal V}} {\pi _i}(k)\sum\limits_{i = 1}^N {{{({x_i}(k) - {\pi ^T}(k)x(k))}^2}} \\
 \ge \frac{1}{2}\mathop {\min }\limits_{i \in {\cal V}} {\pi _i}(k){(\mathop {\max }\limits_{i \in {\cal V}} {x_i}(k) - \mathop {\min }\limits_{i \in {\cal V}} {x_i}(k))^2}\\
 \ge \frac{1}{2}\mathop {\min }\limits_{i \in {\cal V}} {\pi _i}(k)(\mathop {\max }\limits_{i \in {\cal V}} {x_i}(k)\\
 \ge \frac{1}{2}\mathop {\min }\limits_{i \in {\cal V}} {\pi _i}(k)(\mathop {\max }\limits_{i \in {\cal V}} {x_i}(k)
\end{array}
\end{equation}

\textcolor{red}{插入不带编号的公式},只需将equation改成\verb+equation*+
\begin{equation*}
  \begin{array}{l}
V(k) \ge \mathop {\min }\limits_{i \in {\cal V}} {\pi _i}(k)\sum\limits_{i = 1}^N {{{({x_i}(k) - {\pi ^T}(k)x(k))}^2}} \\
 \ge \frac{1}{2}\mathop {\min }\limits_{i \in {\cal V}} {\pi _i}(k){(\mathop {\max }\limits_{i \in {\cal V}} {x_i}(k) - \mathop {\min }\limits_{i \in {\cal V}} {x_i}(k))^2}\\
 \ge \frac{1}{2}\mathop {\min }\limits_{i \in {\cal V}} {\pi _i}(k)(\mathop {\max }\limits_{i \in {\cal V}} {x_i}(k)\\
 \ge \frac{1}{2}\mathop {\min }\limits_{i \in {\cal V}} {\pi _i}(k)(\mathop {\max }\limits_{i \in {\cal V}} {x_i}(k)
\end{array}
\end{equation*}


\subsection{公式对齐}

但是发现以上的公式并不美观,可以进一步进行对齐完善,\textcolor{red}{仔细对比\eqref{system1}公式代码和\eqref{system2}公式代码的区别},主要先删掉\verb+\begin{array}{l}+以及\verb+\end{array}{l}+,然后要在对齐的地方插入$\&$符号并结合\verb+\begin{split}+指令,完成对齐。

\begin{equation}\label{system2}
\begin{split}
V(k) &\ge \mathop {\min }\limits_{i \in {\cal V}} {\pi _i}(k)\sum\limits_{i = 1}^N {{{({x_i}(k) - {\pi ^T}(k)x(k))}^2}} \\
 &\ge \frac{1}{2}\mathop {\min }\limits_{i \in {\cal V}} {\pi _i}(k){(\mathop {\max }\limits_{i \in {\cal V}} {x_i}(k) - \mathop {\min }\limits_{i \in {\cal V}} {x_i}(k))^2}\\
 &\ge \frac{1}{2}\mathop {\min }\limits_{i \in {\cal V}} {\pi _i}(k)(\mathop {\max }\limits_{i \in {\cal V}} {x_i}(k)\\
 &\ge \frac{1}{2}\mathop {\min }\limits_{i \in {\cal V}} {\pi _i}(k)(\mathop {\max }\limits_{i \in {\cal V}} {x_i}(k).
\end{split}
\end{equation}

\textcolor{red}{公式太长的情形,一行放不下的公式,可参考以下进行修改(参考源latex代码进行区分二者的区别)。}举例1如下,下面第一个式子是直接从mathtype复制,第二个式子插入了标签同时进行了对齐(关键看式中的\&符号插入位置和符号$\backslash$$\backslash$的关系)\verb+\hspace{0.3cm}+来表示对齐时空0.3cm


\[\begin{array}{l}
V(k) \ge \mathop {\min }\limits_{i \in {\cal V}} {\pi _i}(k)\sum\limits_{i = 1}^N {{{({x_i}(k) - {\pi ^T}(k)x(k))}^2}} {\rm{ + }}\frac{1}{2}\mathop {\min }\limits_{i \in {\cal V}} {\pi _i}(k){(\mathop {\max }\limits_{i \in {\cal V}} {x_i}(k) - \mathop {\min }\limits_{i \in {\cal V}} {x_i}(k))^2}\\
{\rm{ + }}\frac{1}{2}\mathop {\min }\limits_{i \in {\cal V}} {\pi _i}(k){(\mathop {\max }\limits_{i \in {\cal V}} {x_i}(k) - \mathop {\min }\limits_{i \in {\cal V}} {x_i}(k))^2}
\end{array}\]

\begin{equation}\label{system3}
  \begin{split}
V(k) \ge& \mathop {\min }\limits_{i \in {\cal V}} {\pi _i}(k)\sum\limits_{i = 1}^N {{{({x_i}(k) - {\pi ^T}(k)x(k))}^2}} {\rm{ + }}\frac{1}{2}\mathop {\min }\limits_{i \in {\cal V}} {\pi _i}(k){(\mathop {\max }\limits_{i \in {\cal V}} {x_i}(k) - \mathop {\min }\limits_{i \in {\cal V}} {x_i}(k))^2}\\
&\hspace{0.3cm}{\rm{ + }}\frac{1}{2}\mathop {\min }\limits_{i \in {\cal V}} {\pi _i}(k){(\mathop {\max }\limits_{i \in {\cal V}} {x_i}(k) - \mathop {\min }\limits_{i \in {\cal V}} {x_i}(k))^2}
  \end{split}
\end{equation}

举例2如下
\[\begin{array}{l}
V(k) \ge \mathop {\min }\limits_{i \in {\cal V}} {\pi _i}(k)\sum\limits_{i = 1}^N {{{({x_i}(k) - {\pi ^T}(k)x(k))}^2}} {\rm{ + }}\frac{1}{2}\mathop {\min }\limits_{i \in {\cal V}} {\pi _i}(k){(\mathop {\max }\limits_{i \in {\cal V}} {x_i}(k) - \mathop {\min }\limits_{i \in {\cal V}} {x_i}(k))^2}\\
{\rm{ + }}\frac{1}{2}\mathop {\min }\limits_{i \in {\cal V}} {\pi _i}(k){(\mathop {\max }\limits_{i \in {\cal V}} {x_i}(k) - \mathop {\min }\limits_{i \in {\cal V}} {x_i}(k))^2}\\
 \ge \frac{1}{2}\mathop {\min }\limits_{i \in {\cal V}} {\pi _i}(k){(\mathop {\max }\limits_{i \in {\cal V}} {x_i}(k) - \mathop {\min }\limits_{i \in {\cal V}} {x_i}(k))^2}\\
 \ge \frac{1}{2}\mathop {\min }\limits_{i \in {\cal V}} {\pi _i}(k)(\mathop {\max }\limits_{i \in {\cal V}} {x_i}(k)
\end{array}\]




\begin{equation}\label{system4}
  \begin{split}
V(k) &\ge \mathop {\min }\limits_{i \in {\cal V}} {\pi _i}(k)\sum\limits_{i = 1}^N {{{({x_i}(k) - {\pi ^T}(k)x(k))}^2}} {\rm{ + }}\frac{1}{2}\mathop {\min }\limits_{i \in {\cal V}} {\pi _i}(k){(\mathop {\max }\limits_{i \in {\cal V}} {x_i}(k) - \mathop {\min }\limits_{i \in {\cal V}} {x_i}(k))^2}\\
&\hspace{2cm}{\rm{ + }}\frac{1}{2}\mathop {\min }\limits_{i \in {\cal V}} {\pi _i}(k){(\mathop {\max }\limits_{i \in {\cal V}} {x_i}(k) - \mathop {\min }\limits_{i \in {\cal V}} {x_i}(k))^2}\\
& \ge \frac{1}{2}\mathop {\min }\limits_{i \in {\cal V}} {\pi _i}(k){(\mathop {\max }\limits_{i \in {\cal V}} {x_i}(k) - \mathop {\min }\limits_{i \in {\cal V}} {x_i}(k))^2}\\
 &\ge \frac{1}{2}\mathop {\min }\limits_{i \in {\cal V}} {\pi _i}(k)(\mathop {\max }\limits_{i \in {\cal V}} {x_i}(k)
  \end{split}
\end{equation}


\subsection{定理环境}

定理插入可参考如下
\begin{theorem}
设$f$在凸集$D \subset {R^n}$上一阶连续可微,则
\begin{itemize}
\item $f$在$D$上为凸函数的充要条件是
\begin{equation*}
f(x) \ge f({x^*}) + \nabla f{({x^*})^T}(x - {x^*}),\forall {x^*},x \in D.
\end{equation*}
\item  $f$在$D$上严格凸的充要条件是$x \ne y$时,
\begin{equation*}
f(x) > f({x^*}) + \nabla f{({x^*})^T}(x - {x^*}),\forall {x^*},x \in D.
\end{equation*}
\item $f$在$D$上一致凸的充要条件是,存在常数$c > 0$,使得
成立
\begin{equation*}
f(x) > f({x^*}) + \nabla f{({x^*})^T}(x - {x^*}) + c{\left\| {x - {x^*}} \right\|^2},\forall {x^*},x \in D.
\end{equation*}
\end{itemize}
\end{theorem}

\subsection{定义环境}
\begin{definition}
设集合 $D \subset {R^n}.$ 称集合$D$为凸集, 是指对任意的 $x,y \in {R^n}$及任意
的实数$\lambda  \in [0,1],$ 都有$\lambda x + (1 - \lambda )y \in D.$
\end{definition}

\subsection{假设环境}
\begin{assumption}
设$f$在凸集$D \subset {R^n}$上一阶连续可微。
\end{assumption}
\subsection{问题环境}
\begin{problem}
  设$f$在凸集$D \subset {R^n}$上一阶连续可微。
  \end{problem}
 
\subsection{其它环境}
其它环境可参考下图配置
\textcolor{red}{插入引理、推论等可参考下图对定理环境做对应修改得到(如图\ref{fig_定理环境}所示}。
 \begin{figure}[!htb]
  \centering
  \includegraphics[width=1\textwidth]{定理环境}
  \caption{根据此图做对应修改可插入引理、推论等,具体代码可看latex开头部分环境定义}
  \label{fig_定理环境}
\end{figure}



\subsection{算法设计}

\begin{algorithm}[H]
    \KwIn {西瓜集}
    \KwOut{分类结果}
    初始化\;
    \While{迭代未终止}{
        学习\;
        \eIf{西瓜属性}{
            统计\;
            计算\;
        }{
            下一次迭代\;
        }
    }
    \caption{西瓜集分类算法}
\end{algorithm}
